\documentclass[a4paper,12pt]{report}
\usepackage[margin=2cm]{geometry}
\usepackage[utf8]{inputenc}
\usepackage{listings}
\usepackage{color}
\usepackage{xcolor}
\usepackage{hyperref}
%\usepackage{zeta}
%\usepackage[inline]{trackchanges}
\usepackage{pgf,pgfarrows,pgfnodes,pgfautomata,pgfheaps}
%\usepackage{makeidx}
%\usepackage{tocloft}
%\usepackage{romanian}
\usepackage{pdfpages}
\usepackage{graphicx}
\usepackage{mathtools}
\usepackage{url}
\graphicspath{ {fig/} }

\definecolor{codegreen}{rgb}{0,0.6,0}
\definecolor{codegray}{rgb}{0.5,0.5,0.5}
\definecolor{codepurple}{rgb}{0.58,0,0.82}
\definecolor{backcolour}{rgb}{0.95,0.95,0.92}

\lstdefinestyle{mystyle}{
    backgroundcolor=\color{backcolour},
    commentstyle=\color{codegreen},
    keywordstyle=\color{magenta},
    numberstyle=\tiny\color{codegray},
    stringstyle=\color{codepurple},
    basicstyle=\footnotesize,
    breakatwhitespace=false,
    breaklines=true,
    captionpos=b,
    keepspaces=true,
    numbers=left,
    numbersep=5pt,
    showspaces=false,
    showstringspaces=false,
    showtabs=false,
    tabsize=2
}

\lstset{style=mystyle}

\usepackage[backend=bibtex, style=alphabetic, sorting=ynt]{biblatex}
\addbibresource{BOC_2013.bib}
\begin{document}
\newcommand{\h}{\texttt}

\vspace{-5cm}
\begin{center}
Department of Computer Science\\
Technical University of Cluj-Napoca\\https://www.overleaf.com/project/5c7f87ea86f3ee70d98c7434
\pgfimage[width=10cm]{fig/footer}
\end{center}
\vspace{1cm}
%\maketitle
\begin{center}
\begin{Large}
Knowledge-Based Systems\\
\end{Large}
Laboratory activity\\


Ontology title: \textbf{TES V: Skyrim Skill Tree}\\
Team name: \textbf{Dovahyol}\\
Students: Coman Nicolae\\
          Zavaczki Peter\\
Email: ncoman32@yahoo.com\\
       peter.zavaczki@gmail.com\\

\vspace*{14cm}

Assoc. Prof.dr. eng. Adrian Groza\\
Adrian.Groza@cs.utcluj.ro
\end{center}

\newpage

\tableofcontents
\clearpage
\chapter{Contents}
\section{Competency questions}
Use cases:
\begin{itemize}
 \item Anyone who wants to play the game TES V: Skyrim.
 \item Anyone who wants to know which skills can be learnt at current level.
 \item Anyone who wants to know what perks the skills provide.
 \item Anyone who wants to know suitable skills based on the character's build.
 \item Anyone who wants to know the pre-required skill in order to learn a specific skill.
 \item Anyone who wants to know the level required to learn a specific skill.
 \item Anyone who wants to know the skills not worth prioritizing.
\end{itemize}
\hfill \break
Competency questions:
\begin{itemize}
  \item What are the classes of characters I can play?
  \item What are the skills suitable for build X?
  \item Should I invest in skill tree X if my character is build Y?
  \item What skills can I unlock at level X?
  \item What skill is required for unlocking skill X?
  \item What level is required for unlocking skill X?
  \item What are the perks provided by skill X?
\end{itemize}


\clearpage
\section{Related ontologies}
The ontologies we found were related to ours based on the fact that they all tackle the topic of video games.
\begin{itemize}
    \item Dota 2 ontology - An ontology describing a scenario from the game - \url{https://ontohub.org/repositories/dota-2-ontology}
    \item Core Game Ontology - An ontology classifying games by their properties - \url{http://autosemanticgame.institutedigitalgames.com/ontologies/core-game-ontology/}
    \item Dota 2 item ontology - An ontology about the items and builds in Dota 2 - \url{https://ontohub.org/boc2018/Dota\%202\%20Item\%20ontology}
\end{itemize}

Unfortunately none of these ontologies are useful to us, as we tackle a very specific topic. None of them will be used.


\clearpage
\section{Tbox}


\clearpage
\section{Abox}


\clearpage
\section{Rules}
In our ontology we defined one rule to sort skills into the \textit{NOT\_UPGRADEABLE} category. This was necessary since our skills can only belong to one of the two categories: \textit{UPGRADEABLE} or \textit{NOT\_UPGRADEABLE}. Using this rule made it easier for us to build the ontology since out of the 180 skills only 27 are upgradeable. After defining the rule, we run it to activate it.
\begin{lstlisting}
(define-rule (?x NOT_UPGRADEABLE) (and (?x SKILL) (neg (?x UPGRADEABLE))))

(run-all-rules)  
\end{lstlisting}

\clearpage
\section{Queries}
The last thing we do as part of loading the ontology and before we run our queries is running all the rules with the \textit{run-all-rules} command. After this step, we can run our evaluation queries.
We check the consistency of our ontology with the following queries.
\begin{lstlisting}
(abox-consistent?)
(tbox-cyclic?)
(tbox-coherent?)

(realize-abox)
(classify-tbox)
\end{lstlisting}
Then we check the size of our ontology with the following queries.
\begin{lstlisting}
(evaluate (length (all-individuals)))
(evaluate (length (all-atomic-concepts)))
(evaluate (length (all-roles)))
(evaluate (length (all-rules)))

(all-concept-assertions)
(all-role-assertions)
(all-constraints)

(describe-tbox)
(describe-abox)

(taxonomy)
\end{lstlisting}
Then we check the expressivity of our ontology with the following queries.
\begin{lstlisting}
(get-tbox-language)
(get-abox-language)

(all-features)
(all-transitive-roles)
\end{lstlisting}
Finally we answer some of the competency question we extracted previously with the following queries.
\begin{itemize}
  \item What are the classes of characters I can play?
  \begin{lstlisting}
  ; --------- REPLACE THIS LINE WITH ANSWER QUERY ---------
  \end{lstlisting}
  \item What are the skills suitable for build X?
  \begin{lstlisting}
  ; --------- REPLACE THIS LINE WITH ANSWER QUERY ---------
  \end{lstlisting}
  \item Should I invest in skill class X if my character is build Y?
  \begin{lstlisting}
  ; --------- REPLACE THIS LINE WITH ANSWER QUERY ---------
  \end{lstlisting}
  \item What skills can I unlock at level X?
  \begin{lstlisting}
  ; --------- REPLACE THIS LINE WITH ANSWER QUERY ---------
  \end{lstlisting}
  \item What skill is required for unlocking skill X?
  \begin{lstlisting}
  ; --------- REPLACE THIS LINE WITH ANSWER QUERY ---------
  \end{lstlisting}
  \item What level is required for unlocking skill X?
  \begin{lstlisting}
  ; --------- REPLACE THIS LINE WITH ANSWER QUERY ---------
  \end{lstlisting}
  \item What are the perks provided by skill X?
  \begin{lstlisting}
  ; --------- REPLACE THIS LINE WITH ANSWER QUERY ---------
  \end{lstlisting}
\end{itemize}


\clearpage
\appendix

\chapter{Original code}
\section{Racer ontology}
\lstinputlisting{../onto.racer}


\section{Racer evaluation}
\lstinputlisting{../evaluation.racer}


%\printbibliography

\vspace{2cm}
\begin{center}
Intelligent Systems Group\\
\pgfimage[width=10cm]{fig/footer}
\end{center}



\end{document}
